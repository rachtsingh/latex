%!TEX program = xelatex
%==============================================================================
% Homework X, CSX (Fall 2015)
%==============================================================================
\newcommand{\thishw}{\bf Homework \#}
\newcommand{\myname}{Your Name}
\newcommand{\mylogin}{<username>@college.harvard.edu}

%==============================================================================
% Formatting parameters.
%==============================================================================

\documentclass[11pt]{article} % 10pt article, want AMS fonts.
\makeatletter					% Make '@' accessible.
\pagestyle{myheadings}				% We do our own page headers.
\def\@oddhead{\bf <CS\#\#\#> - \thishw\hfill \myname\, (\mylogin)} % Here they are.
\def\thesection{Problem\hskip-1em\ }		% Section headlines.
\oddsidemargin=0in				% Left margin minus 1 inch.
\evensidemargin=0in				% Same for even-numbered pages.
\textwidth=6.5in				% Text width (8.5in - margins).
\topmargin=0in					% Top margin minus 1 inch.
\headsep=0.3in					% Distance from header to body.
\textheight=8in					% Body height (incl. footnotes)
\skip\footins=4ex				% Space above first footnote.
\hbadness=10000					% No "underfull hbox" messages.
\makeatother					% Make '@' special again.

%\usepackage{newalg}
\usepackage{amsmath,amsfonts,amssymb}
\usepackage{latexsym}
\usepackage{color}
\usepackage{fontspec}
% \usepackage{minted}
\usepackage{mathtools}
\DeclarePairedDelimiter\ceil{\lceil}{\rceil}
\DeclarePairedDelimiter\floor{\lfloor}{\rfloor}

% \setsansfont{Calibri}
% \setmonofont{Consolas}
% \usepackage{psfig}

%==============================================================================
% Macros.
%==============================================================================
\newcommand{\problem}[1]{\section{#1}}		% Problem.
\newcommand{\new}[1]{{\em #1\/}}		% New term (set in italics).
\newcommand{\set}[1]{\{#1\}}			% Set (as in \set{1,2,3})
\newcommand{\setof}[2]{\{\,{#1}|~{#2}\,\}}	% Set (as in \setof{x}{x > 0})
\newcommand{\C}{\mathbb{C}}	                % Complex numbers.
\newcommand{\N}{\mathbb{N}}                     % Positive integers.
\newcommand{\Q}{\mathbb{Q}}                     % Rationals.
\newcommand{\R}{\mathbb{R}}                     % Reals.
\newcommand{\Z}{\mathbb{Z}}                     % Integers.
\newcommand{\compl}[1]{\overline{#1}}		% Complement of 
\newcommand{\on}{\operatorname}
\newcommand{\Var}{\on{\bf Var}}
\newcommand{\E}{\on{\bf E}}
\newenvironment{statement}{\par\color{red}}{\par\medskip}
% ...            

%==============================================================================
% Title.
%==============================================================================

\begin{document}
\centerline{\LARGE\thishw}

\problem{1.1}
\noindent{(a)} This is an example problem subpart

\smallskip

\noindent{(b)} Here's another problem subpart

\smallskip

\noindent{(c)} Make sure to separate the parts via \texttt{\\smallskip}.

\problem{1.2}

\noindent{(a)} Here's another problem.

\end{document}

% Here's an example of a code sample if you ever need it.
%
% \renewcommand{\theFancyVerbLine}{
%   \sffamily\textcolor[rgb]{0.5,0.5,0.5}{\scriptsize\arabic{FancyVerbLine}}}
%   \begin{minted}[mathescape,
%                linenos,
%                numbersep=5pt,
%                gobble=3,
%                frame=lines,
%                framesep=2mm]{python}

% 	import random
% 	from decimal import Decimal, getcontext, ROUND_FLOOR
% 	getcontext().prec = 50 # high precision unnecessary, but also safer
% 	getcontext().rounding = ROUND_FLOOR # interesting way of rounding from StackOverflow
% 	import matplotlib.pyplot as plt

% 	BIGNUM = 100000000000000 # as big as you want
% 	ITERATIONS = 10000

% 	def get_random_variable(k):
% 	    u = Decimal(random.randrange(BIGNUM))/BIGNUM
% 	    return (((1/(u)) ** (1/Decimal(k))).to_integral_exact())

% 	plots = []

% 	for i in range(7):
% 	    k = 3 # change as necessary
% 	    values = []
% 	    averages = []
% 	    running_sum = 0
% 	    for i in range(ITERATIONS):
% 	        values.append(get_random_variable(k))
% 	        running_sum += values[-1]
% 	        averages.append(running_sum/(i + 1))
	    
% 	    plots.append(averages)

% 	colors = ["green", "blue", "red", "purple", "brown", "yellow", "orange"]
% 	for i in range(7):
% 	    plt.plot(plots[i], color=colors[i])

% \end{minted}